% all-in-one cheatsheet layout (Michael Franzen, 2013)
\documentclass[a4paper]{article}

% geometry settings
\usepackage[top=2cm, bottom=2.5cm, left=2cm, right=2cm]{geometry}

% font settings
%\usepackage[light,math]{kurier}
\usepackage[T1]{fontenc}
\usepackage[utf8]{inputenc}
\usepackage{marvosym}
\usepackage{amssymb}
\usepackage{amsfonts}
\usepackage{amsmath}
\usepackage{amsthm}

% colors
\usepackage{xcolor}
\definecolor{lightgray}{gray}{0.8}

% formatting
\usepackage{paralist}
\usepackage{multicol}
\usepackage{tabularx}
\usepackage{Tabbing}
\usepackage{booktabs}
\usepackage{fancyhdr}
\usepackage{url}
\usepackage[framemethod=tikz]{mdframed}
\pagestyle{fancy}

% math
\usepackage{array}
\usepackage{eqnarray}
\usepackage{mathtools}
\usepackage{cases}

% figures
\usepackage{wrapfig}
\usepackage{subfigure}

% figure modules
\usepackage{graphicx}
\usepackage{tikz}
\usetikzlibrary{positioning,calc, shapes}
\usepackage{algorithm2e}
\usepackage{verbatim}	

% TOC & Glossary
\usepackage{sectsty}
\usepackage[nottoc,notlof,notlot]{tocbibind}
\usepackage[titles,subfigure]{tocloft}

% commands
\usepackage{xargs}
\usepackage{ifthen}

% head line
\fancyhf{}
\chead{Graph Theory - Sheet 9 - \today\\J. Batzill (1698622), M. Franzen (1696933), J. Labeit (1656460)}
\renewcommand{\headrulewidth}{0.4pt} %obere Trennlinie

\newcommand{\sheetnumber}{1}
\newcommand{\unaryminus}{\scalebox{0.5}[1.0]{\( - \)}}

% (problem number)
\surroundwithmdframed[
    hidealllines=true,
    backgroundcolor=gray!10,
    skipbelow=\baselineskip,
    skipabove=\baselineskip
]{mylemma}

\surroundwithmdframed[
	linecolor=white,
	skipbelow=\baselineskip,
	skipabove=\baselineskip
]{mytheorem}

\tikzstyle{nod}= [circle, draw,inner sep=0pt, minimum size=0.5cm] 

\begin{document}
	
	\newtheorem{mytheorem}{Theorem}[section]
	\newtheorem{mylemma}{Lemma}[mytheorem]	

	\newenvironmentx*{solution}[1]{\section*{Problem #1}\addtocounter{section}{1}\setcounter{mylemma}{0}\setcounter{mytheorem}{0}}{}
	\newenvironmentx*{theorem}[1]{\begin{mytheorem}#1\begin{proof}}{\end{proof}\end{mytheorem}}
	\newenvironmentx*{lemma}[1]{\begin{mylemma}#1\begin{proof}}{\end{proof}\end{mylemma}}
	
	
		\begin{solution}{34}
		In the following we will first proof a lemma and then the theorem. 
		
		\begin{lemma}{In a triangle-free graph $G=(V,E)$ with $|V|=n$ if there is a vertex $v \in V$ with $deg(v) = a$, then all the neighbours of $v$ have at most a degree of $n-a$}
			Let $u \in V$ be any neighbour of $v$. 
			$u$ cannot be connected to nodes adjacent to $v$ or we would have a triangle with $u$ and $v$. 
			So $deg(u) \leq |V - neighbourhood(v)| = n - deg(v) \leq n-a$. This line of argumentation can be used on all neighbours of $v$. 
		\end{lemma}
				
			
		\begin{theorem}{For every triangle-free graph $G=(V,E)$ with $|V|=n$ the inequality $\sum_{v \in V}{deg(v)^2 \leq \frac{n^3}{4}}$ holds.}
			Let $v \in V$ be the vertex with the highest degree in $G$, $deg(v) = a$. \\
			Using the lemma we know that for all $u \in neighbours(v) : deg(u) \leq n-a$. Now we now that we have $a$ vertices with maximum degree $n-a$ and the rest $n-a$ vertices have maximum degree $a$. 
			This gives us the upper bound for the formula $\sum_{v \in V}{deg(v)^2} \leq a*(n-a)^2 + (n-a)*a^2$ with the parameter $a$. 
			Now we can seek for the maximum of the function to get a precise upper bound $f(a) = a*(n-a)^2 + (n-a)*a^2 = a*n^2-a^2*n$.
			$f'(a) = n^2 - 2*a*n = n(n-2a)$. So $f$ has a maximum at $\frac{n}{2}$, so to achieve the maximum for  $\sum_{v \in V}{deg(v)^2}$ it is best for all the vertices to have degree $\frac{n}{2}$.  
			Now we easily get the upper bound  $\sum_{v \in V}{deg(v)^2 \leq f(a) \leq f(\frac{n}{2}) = \frac{n^3}{4}}$. 
		\end{theorem}
	\end{solution}
	
	
\end{document}
