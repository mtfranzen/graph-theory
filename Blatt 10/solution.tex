% all-in-one cheatsheet layout (Michael Franzen, 2013)
\documentclass[a4paper]{article}

% geometry settings
\usepackage[top=2cm, bottom=2.5cm, left=2cm, right=2cm]{geometry}

% font settings
%\usepackage[light,math]{kurier}
\usepackage[T1]{fontenc}
\usepackage[utf8]{inputenc}
\usepackage{marvosym}
\usepackage{amssymb}
\usepackage{amsfonts}
\usepackage{amsmath}
\usepackage{amsthm}

% colors
\usepackage{xcolor}
\definecolor{lightgray}{gray}{0.8}

% formatting
\usepackage{paralist}
\usepackage{multicol}
\usepackage{tabularx}
\usepackage{Tabbing}
\usepackage{booktabs}
\usepackage{fancyhdr}
\usepackage{url}
\usepackage[framemethod=tikz]{mdframed}
\pagestyle{fancy}

% math
\usepackage{array}
\usepackage{eqnarray}
\usepackage{mathtools}
\usepackage{cases}

% figures
\usepackage{wrapfig}
\usepackage{subfigure}

% figure modules
\usepackage{graphicx}
\usepackage{tikz}
\usetikzlibrary{positioning,calc, shapes}
\usepackage{algorithm2e}
\usepackage{verbatim}	

% TOC & Glossary
\usepackage{sectsty}
\usepackage[nottoc,notlof,notlot]{tocbibind}
\usepackage[titles,subfigure]{tocloft}

% commands
\usepackage{xargs}
\usepackage{ifthen}

% head line
\fancyhf{}
\chead{Graph Theory - Sheet 10 - \today\\J. Batzill (1698622), M. Franzen (1696933), J. Labeit (1656460)}
\renewcommand{\headrulewidth}{0.4pt} %obere Trennlinie

\newcommand{\sheetnumber}{1}
\newcommand{\unaryminus}{\scalebox{0.5}[1.0]{\( - \)}}

% (problem number)
\surroundwithmdframed[
    hidealllines=true,
    backgroundcolor=gray!10,
    skipbelow=\baselineskip,
    skipabove=\baselineskip
]{mylemma}

\surroundwithmdframed[
	linecolor=white,
	skipbelow=\baselineskip,
	skipabove=\baselineskip
]{mytheorem}

\tikzstyle{nod}= [circle, draw,inner sep=0pt, minimum size=0.5cm] 

\begin{document}
	
	\newtheorem{mytheorem}{Theorem}[section]
	\newtheorem{mylemma}{Lemma}[mytheorem]	

	\newenvironmentx*{solution}[1]{\section*{Problem #1}\addtocounter{section}{1}\setcounter{mylemma}{0}\setcounter{mytheorem}{0}}{}
	\newenvironmentx*{theorem}[1]{\begin{mytheorem}#1\begin{proof}}{\end{proof}\end{mytheorem}}
	\newenvironmentx*{lemma}[1]{\begin{mylemma}#1\begin{proof}}{\end{proof}\end{mylemma}}
	
	
	\begin{solution}{37}
		\begin{lemma}{For every edge $e=(u,v)$ in a graph $G$ without $C_6$ there are at most 2 $C_4$ in G which contain e.}
			
		\end{lemma} 
		By using the lemma we can simply construct $H$ by removing edges from $G$. 
		\begin{theorem}{Every graph $G$ without $C_6$ has a subgraph $H$ with $|E(H)| \geq \frac{|E(G)|}{2}$, which contains $C_4$.} 
			Let $G$ be such a graph. 
			We know by using the lemma that evere edge is at most part of 2 $C_4$s. 
			Let's denote $k$ with the number of $C_4$s in G.
			$k \leq \frac{2}{4} * |E(G)| = \frac{|E(G)|}{2}$, because every edge is at most in 2 $C_4$s and $|C_4| = 4$.  
			Now we can construct $H$ by removing an edge from every $C_4$ in G. 
			It is easy to see that by removing an edge no new cycles are created, hence $H$ contains no $C_4$ and $|E(H)| = |E(G)| - k \geq |E(G)| - \frac{|E(G)|}{2} \leq \frac{|E(G)|}{2}$. 
		\end{theorem}
	\end{solution}
	
	
\end{document}
