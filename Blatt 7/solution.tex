% all-in-one cheatsheet layout (Michael Franzen, 2013)
\documentclass[a4paper]{article}

% geometry settings
\usepackage[top=2cm, bottom=2.5cm, left=2cm, right=2cm]{geometry}

% font settings
%\usepackage[light,math]{kurier}
\usepackage[T1]{fontenc}
\usepackage[utf8]{inputenc}
\usepackage{marvosym}
\usepackage{amssymb}
\usepackage{amsfonts}
\usepackage{amsmath}
\usepackage{amsthm}

% colors
\usepackage{xcolor}
\definecolor{lightgray}{gray}{0.8}

% formatting
\usepackage{paralist}
\usepackage{multicol}
\usepackage{tabularx}
\usepackage{Tabbing}
\usepackage{booktabs}
\usepackage{fancyhdr}
\usepackage{url}
\usepackage[framemethod=tikz]{mdframed}
\pagestyle{fancy}

% math
\usepackage{array}
\usepackage{eqnarray}
\usepackage{mathtools}
\usepackage{cases}

% figures
\usepackage{wrapfig}
\usepackage{subfigure}

% figure modules
\usepackage{graphicx}
\usepackage{tikz}
\usetikzlibrary{positioning,calc, shapes}
\usepackage{algorithm2e}
\usepackage{verbatim}	

% TOC & Glossary
\usepackage{sectsty}
\usepackage[nottoc,notlof,notlot]{tocbibind}
\usepackage[titles,subfigure]{tocloft}

% commands
\usepackage{xargs}
\usepackage{ifthen}

% head line
\fancyhf{}
\chead{Graph Theory - Sheet 7 - \today\\J. Batzill (1698622), M. Franzen (1696933), J. Labeit (1656460)}
\renewcommand{\headrulewidth}{0.4pt} %obere Trennlinie

\newcommand{\sheetnumber}{1}

% (problem number)
\surroundwithmdframed[
    hidealllines=true,
    backgroundcolor=gray!10,
    skipbelow=\baselineskip,
    skipabove=\baselineskip
]{mylemma}

\surroundwithmdframed[
	linecolor=white,
	skipbelow=\baselineskip,
	skipabove=\baselineskip
]{mytheorem}

\tikzstyle{nod}= [circle, draw,inner sep=0pt, minimum size=0.5cm] 

\begin{document}
	
	\newtheorem{mytheorem}{Theorem}[section]
	\newtheorem{mylemma}{Lemma}[mytheorem]	

	\newenvironmentx*{solution}[1]{\section*{Problem #1}\addtocounter{section}{1}\setcounter{mylemma}{0}\setcounter{mytheorem}{0}}{}
	\newenvironmentx*{theorem}[1]{\begin{mytheorem}#1\begin{proof}}{\end{proof}\end{mytheorem}}
	\newenvironmentx*{lemma}[1]{\begin{mylemma}#1\begin{proof}}{\end{proof}\end{mylemma}}


	\begin{solution}{27}
		In the following we will show that for every Graph G on $n$ vertices we have \\ 
		$\chi (G) + \chi (\bar{G}) \geq max (\omega (G) , \frac{n}{\alpha(G)}) + max (\omega (\bar{G}) , \frac{n}{\alpha(\bar{G})}) \geq 2\sqrt{n}$. \\
		Using the lemma of small colouring results we know that $\chi (G)  \geq max (\omega (G) , \frac{n}{\alpha(G)})$ and that
		$\chi (\bar{G})  \geq max (\omega (\bar{G}) , \frac{n}{\alpha(\bar{G})})$. 
		Hence, $\chi (G) + \chi (\bar{G}) \geq max (\omega (G) , \frac{n}{\alpha(G)}) + max (\omega (\bar{G}) , \frac{n}{\alpha(\bar{G})}) =  max (\omega (G) , \frac{n}{\alpha(G)}) + max (\alpha (G) , \frac{n}{\omega(G)}) \geq \omega (G) + \frac{n}{\omega(G)} \geq \sqrt{n} +  \frac{n}{\sqrt{n}} = 2\sqrt{n}$ \\
		The inequality $\omega (G) + \frac{n}{\omega(G)} \geq \sqrt{n} +  \frac{n}{\sqrt{n}}$ is true for $\omega(G) \in [1,n]$ because the function $f(x) = x + \frac{n}{x}$ has a local minima at $x=\sqrt{n}$ in the interval $x \in [1,n]$. 
	\end{solution}


		
\end{document}
