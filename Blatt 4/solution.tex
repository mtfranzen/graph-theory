% all-in-one cheatsheet layout (Michael Franzen, 2013)
\documentclass[a4paper]{article}

% geometry settings
\usepackage[top=2cm, bottom=2.5cm, left=2cm, right=2cm]{geometry}

% font settings
%\usepackage[light,math]{kurier}
\usepackage[T1]{fontenc}
\usepackage[utf8]{inputenc}
\usepackage{marvosym}
\usepackage{amssymb}
\usepackage{amsfonts}
\usepackage{amsmath}
\usepackage{amsthm}

% colors
\usepackage{xcolor}
\definecolor{lightgray}{gray}{0.8}

% formatting
\usepackage{paralist}
\usepackage{multicol}
\usepackage{tabularx}
\usepackage{Tabbing}
\usepackage{booktabs}
\usepackage{fancyhdr}
\usepackage{url}
\usepackage[framemethod=tikz]{mdframed}
\pagestyle{fancy}

% math
\usepackage{array}
\usepackage{eqnarray}
\usepackage{mathtools}
\usepackage{cases}

% figures
\usepackage{wrapfig}
\usepackage{subfigure}

% figure modules
\usepackage{graphicx}
\usepackage{tikz}
\usetikzlibrary{positioning,calc, shapes}
\usepackage{algorithm2e}
\usepackage{verbatim}	

% TOC & Glossary
\usepackage{sectsty}
\usepackage[nottoc,notlof,notlot]{tocbibind}
\usepackage[titles,subfigure]{tocloft}

% commands
\usepackage{xargs}
\usepackage{ifthen}

% head line
\fancyhf{}
\chead{Graph Theory - Sheet 4 - \today\\J. Batzill (1698622), M. Franzen (1696933), J. Labeit (1656460)}
\renewcommand{\headrulewidth}{0.4pt} %obere Trennlinie

\newcommand{\sheetnumber}{1}

% (problem number)
\surroundwithmdframed[
    hidealllines=true,
    backgroundcolor=gray!10,
    skipbelow=\baselineskip,
    skipabove=\baselineskip
]{mylemma}

\surroundwithmdframed[
	linecolor=white,
	skipbelow=\baselineskip,
	skipabove=\baselineskip
]{mytheorem}

\tikzstyle{nod}= [circle, draw,inner sep=0pt, minimum size=0.5cm] 

\begin{document}
	
	\newtheorem{mytheorem}{Theorem}[section]
	\newtheorem{mylemma}{Lemma}[mytheorem]	

	\newenvironmentx*{solution}[1]{\section*{Problem #1}\addtocounter{section}{1}\setcounter{mylemma}{0}\setcounter{mytheorem}{0}}{}
	\newenvironmentx*{theorem}[1]{\begin{mytheorem}#1\begin{proof}}{\end{proof}\end{mytheorem}}
	\newenvironmentx*{lemma}[1]{\begin{mylemma}#1\begin{proof}}{\end{proof}\end{mylemma}}


	\begin{solution}{13}
		\begin{theorem}{If a graph has an ear-decomposition, then it is $2$-connected.}
			By \emph{Menger's Theorem}, a graph $G$ is $k$-connected if and only if for any two vertices $a, b$ in $G$ there exist $k$ independent $a$-$b$-paths. We find those $2$ paths for any ear-composable graph.
		\end{theorem}
	\end{solution}
	\newpage
	\begin{solution}{14}
		For $0 < l < m \leq d$, we will construct a graph $F(d, l, m)$.\\

		\begin{figure}[h]
			\begin{tikzpicture}
				\node[cloud, cloud puffs=10, cloud puff arc=120, aspect=2, minimum width=6cm, minimum height=4cm, draw, dashed] (V) at (0, 0) {$K_{d+1}$};
				\node[circle, draw] (v1) at (1.5,0.75) {$v_1$};
				\node (vdots) at (1.5,0.125) {$\vdots$};
				\node[circle, draw] (vm) at (1.5, -0.75) {$v_m$};

				\node[cloud, cloud puffs=10, cloud puff arc=120, aspect=2, minimum width=6cm, minimum height=4cm, draw, dashed] (W) at (11, 0) {$K_{d+1}$};
				\node[circle, draw] (w1) at (9.5,0.75) {$w_1$};
				\node (vdots) at (9.5,0.125) {$\vdots$};
				\node[circle, draw] (wm) at (9.5, -0.75) {$w_l$};

				\node[cloud, cloud puffs=10, cloud puff arc=120, aspect=2, minimum width=1.5cm, minimum height=1cm] (S) at (5.5, 0) {$\cdots$};


				\draw[dotted] (v1) -- (S);
				\draw[dotted] (vm) -- (S);
				\draw[dotted] (w1) -- (S);
				\draw[dotted] (wm) -- (S);
			\end{tikzpicture}
			\caption{$F(d, l, m)$}
		\end{figure}
		First, we construct two complete graphs on $d+1$ vertices. $(V, E) \simeq K_{d+1}$, $(W, E') \simeq K_{d+1}$.\\

		Then, we join $m$ vertices $v_1, ..., v_m \in V$ of the first complete graph and $l$ vertices $w_1, ..., w_l \in W$ of the second such that each $v_i$ has a degree of exactly $d+1$ and each $w_j$ of at least $d+1$ $(i \in [m], j \in [l])$.\\

		Formally, for our constructed graph $F(d, l, m) := (V_F, E_F)$,  the vertex set is the union of both complete graphs ($V_F = V \cup W$) and it's edge set is defined by
		\begin{equation}
			E_F = E \cup E' \cup \{\{v_i, w_j\}\ |\ \delta_{ij} = 1\ (i, j \in \mathbb{N})\}
		\end{equation}
		
		for a delta function $\delta_{ij}$ ($i, j \in \mathbb{N}$)
		\begin{equation}
			\delta_{ij} =
					\begin{cases}
						1 & \text{if } i = j \text{ or }  i > l, j = l\\
						0 & \text{otherwise}
					\end{cases}
		\end{equation}\\
		We will show that
		\begin{compactitem}%
			\item $\delta(F(d,l,m)) = d$%
			\item $\kappa(F(d,l,m)) = l$%
			\item $\kappa'(F(d,l,m)) = m$%
		\end{compactitem}\ \\
		$\mathbf{\delta(F(d,l,m)) = d}$\\
		No degree of a vertex of the complete graphs has been decreased. \emph{Thus, $\delta(F(d,l,m)) \geq \delta(K_{d+1}) = d$.}

		Moreover, we have increased the degree of exactly $l + m < 2(d+1)$ vertices.  Indeed, the complete graph on $d+1$ vertices is $d$-regular and hence there is at least one vertex of degree $d$ in $F(d,l,m)$. \emph{Thus, $\delta(F(d,l,m)) \leq d$.}

		From these considerations, $\delta(F(d,l,m)) = d$.\ \\ \ \\
		$\mathbf{\kappa(F(d,l,m)) = l}$\\
		In $F(d,l,m)$, the two complete graphs are only joined by edges between $l$ vertices of one and $m$ vertices of another complete graph.
		The graph obviously disconnects by removing those first $l$ vertices. \emph{Thus, $\kappa(F(d,l,m)) \leq l$.}

		Moreover, a complete graph on $d+1$ vertices is internally connected with $\kappa(K_{d+1}) = d > l$. Hence, if we found a subset of $l' < l$ vertices that disconnects $F(d, l, m)$, it had to consist of the complete graphs' vertices that we have affected in our construction. However, between the complete graphs there are $l$ edges not sharing an endpoint.

		\emph{Thus, it is neither possible to disconnect one of the complete graph by removing less than $l$ vertices nor is it possible to remove the inter-connection between the two complete graphs by removing less than $l$ vertices.}

		From these considerations, $\kappa(F(d,l,m)) = l$.\ \\ \ \\
		$\mathbf{\kappa'(F(d,l,m)) = m}$\\
		In $F(d,l,m)$, the two complete graphs are only joined by exactly $m$ edges and a removal of those $m$ vertices obviously disconnects $F(d, l, m)$. \emph{Thus, $\kappa'(F(d,l,m)) \leq m$}.

		Moreover, a complete graph on $d+1$ vertices is internally connected with $\kappa'(K_{d+1}) = d > m$.

		\emph{Thus, it is neither possible to disconnect one of the complete graph by removing less than $m$ edges nor is it possible to remove the inter-connection between the two complete graphs by removing less than $m$ edges.}

		From these considerations, $\kappa'(F(d,l,m)) = m$.
	\end{solution}
	\newpage
	\begin{solution}{15}
		I will prove that any block-cut-vertex graph is a tree, by showing by contradiction that any block-cut-vertex graph is acyclic and connected. 
		\begin{theorem}{The block-cut-vertex graph $G=(V,E)$ of any connected graph $G'=(V',E')$ is a tree.}
		Let's assume for the sake of contradiction that $G$ has a cycle $C=(b_1b_2...b_1)$.
		Let's denote the subgraphs $B_1,B_2,...B_n$ of $G'$ which are the 2-connected components and bridges corresponding to the nodes $b_1,b_2,...b_n$ of $G$. 
		Let $B_1$ and $B_2$ be as stated above two different subgraphs of $G'$. 
		Because the corresponding nodes $b_1$ and $b_2$ are adjacent in $G$, $B_1$ and $B_2$ have to share a vertex $x \in V(B_1) \cap V(B_2)$. 
		We can use the same argument for each pair $B_i, B_{i+1}$. 
		Additionally, we know because each component $B_j$ is either 2-connected or a bridge. 
		Thus we can find a circle through all the components $B_1, B_2...B_n$ which is 2-connected.
		this is a contradiction to $B_1, B_2...B_n$ being the blocks of an block-cut-vertex graph, because by definition these blocks are either bridges or maximal 2-connected components.
		\end{theorem}
	\end{solution}
	\newpage
	\begin{solution}{16}
	\end{solution}
	
\end{document}
