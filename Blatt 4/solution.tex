% all-in-one cheatsheet layout (Michael Franzen, 2013)
\documentclass[a4paper]{article}

% geometry settings
\usepackage[top=2cm, bottom=2.5cm, left=2cm, right=2cm]{geometry}

% font settings
%\usepackage[light,math]{kurier}
\usepackage[T1]{fontenc}
\usepackage[utf8]{inputenc}
\usepackage{marvosym}
\usepackage{amssymb}
\usepackage{amsfonts}
\usepackage{amsmath}
\usepackage{amsthm}

% colors
\usepackage{xcolor}
\definecolor{lightgray}{gray}{0.8}

% formatting
\usepackage{paralist}
\usepackage{multicol}
\usepackage{tabularx}
\usepackage{Tabbing}
\usepackage{booktabs}
\usepackage{fancyhdr}
\usepackage{url}
\usepackage[framemethod=tikz]{mdframed}
\pagestyle{fancy}

% math
\usepackage{array}
\usepackage{eqnarray}
\usepackage{mathtools}

% figures
\usepackage{wrapfig}
\usepackage{subfig}

% figure modules
\usepackage{graphicx}
\usepackage{tikz}
\usetikzlibrary{positioning,calc, shapes}
\usepackage{algorithm2e}
\usepackage{verbatim}	

% TOC & Glossary
\usepackage{sectsty}
\usepackage[nottoc,notlof,notlot]{tocbibind}
\usepackage[titles,subfigure]{tocloft}

% commands
\usepackage{xargs}
\usepackage{ifthen}

% head line
\fancyhf{}
\chead{Graph Theory - Sheet 4 - \today\\J. Batzill (1698622), M. Franzen (1696933), J. Labeit (1656460)}
\renewcommand{\headrulewidth}{0.4pt} %obere Trennlinie

\newcommand{\sheetnumber}{1}

% (problem number)
\surroundwithmdframed[
    hidealllines=true,
    backgroundcolor=gray!10,
    skipbelow=\baselineskip,
    skipabove=\baselineskip
]{mylemma}

\surroundwithmdframed[
	linecolor=white,
	skipbelow=\baselineskip,
	skipabove=\baselineskip
]{mytheorem}

\tikzstyle{nod}= [circle, draw,inner sep=0pt, minimum size=0.5cm] 

\begin{document}
	
	\newtheorem{mytheorem}{Theorem}[section]
	\newtheorem{mylemma}{Lemma}[mytheorem]	

	\newenvironmentx*{solution}[1]{\section*{Problem #1}\addtocounter{section}{1}\setcounter{mylemma}{0}\setcounter{mytheorem}{0}}{}
	\newenvironmentx*{theorem}[1]{\begin{mytheorem}#1\\\begin{proof}}{\end{proof}\end{mytheorem}}
	\newenvironmentx*{lemma}[1]{\begin{mylemma}#1\\\begin{proof}}{\end{proof}\end{mylemma}}


	\begin{solution}{13}
		\begin{theorem}{If a graph has an ear-decomposition, then it is $2$-connected.}
			By \emph{Menger's Theorem}, a graph $G$ is $k$-connected if and only if for any two vertices $a, b$ in $G$ there exist $k$ independent $a$-$b$-paths. We find those $2$ paths for any ear-composable graph.
		\end{theorem}
	\end{solution}
	\newpage
	\begin{solution}{14}
		For $0 < l < m \leq d$, we will construct a graph $F(d, l, m)$ with 
		\begin{itemize}
			\item $\delta(F(d,l,m)) = d$
			\item $\kappa(F(d,l,m)) = l$
			\item $\kappa'(F(d,l,m)) = m$
		\end{itemize}
		Let $F'( l, m) := K_{l, m}$ be the complete bipartite graph of $l, m$ partition sizes. Thus,
			\begin{equation}
				\kappa(F'(l, m)) = \kappa(K_{l, m}) = \min\{l, m\} = l
			\end{equation}
		and furthermore, 
			\begin{equation}
				\kappa'(F'(l, m)) = \kappa'(K_{l, m}) = \max\{l, m\} = m
			\end{equation}
		.

		Now, we construct  $F(d, l, m)$ by substituting each vertex of $F'(l, m)$ with the complete graph $K_{d}$. Hence, $\delta(F(d,l,m)) = d$. Since $\kappa(K_{d}) = d \geq m$ and $\kappa'(K_{d}) > l$ and since we do not modify the connections between those substituted vertices, the edge- and vertex-connectivity of $F'(l, m)$ are maintained in $F(d, l, m)$.
	\end{solution}
	\newpage
	\begin{solution}{15}
	\end{solution}
	\newpage
	\begin{solution}{16}
	\end{solution}
	
\end{document}
