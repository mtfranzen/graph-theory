% all-in-one cheatsheet layout (Michael Franzen, 2013)
\documentclass[a4paper]{article}

% geometry settings
\usepackage[top=2cm, bottom=2.5cm, left=2cm, right=2cm]{geometry}

% font settings
%\usepackage[light,math]{kurier}
\usepackage[T1]{fontenc}
\usepackage[utf8]{inputenc}
\usepackage{marvosym}
\usepackage{amssymb}
\usepackage{amsfonts}
\usepackage{amsmath}
\usepackage{amsthm}

% colors
\usepackage{xcolor}
\definecolor{lightgray}{gray}{0.8}

% formatting
\usepackage{paralist}
\usepackage{multicol}
\usepackage{tabularx}
\usepackage{Tabbing}
\usepackage{booktabs}
\usepackage{fancyhdr}
\usepackage{url}
\usepackage[framemethod=tikz]{mdframed}
\pagestyle{fancy}

% math
\usepackage{array}
\usepackage{eqnarray}
\usepackage{mathtools}
\usepackage{cases}

% figures
\usepackage{wrapfig}
\usepackage{subfigure}

% figure modules
\usepackage{graphicx}
\usepackage{tikz}
\usetikzlibrary{positioning,calc, shapes}
\usepackage{algorithm2e}
\usepackage{verbatim}	

% TOC & Glossary
\usepackage{sectsty}
\usepackage[nottoc,notlof,notlot]{tocbibind}
\usepackage[titles,subfigure]{tocloft}

% commands
\usepackage{xargs}
\usepackage{ifthen}

% head line
\fancyhf{}
\chead{Graph Theory - Sheet 5 - \today\\J. Batzill (1698622), M. Franzen (1696933), J. Labeit (1656460)}
\renewcommand{\headrulewidth}{0.4pt} %obere Trennlinie

\newcommand{\sheetnumber}{1}

% (problem number)
\surroundwithmdframed[
    hidealllines=true,
    backgroundcolor=gray!10,
    skipbelow=\baselineskip,
    skipabove=\baselineskip
]{mylemma}

\surroundwithmdframed[
	linecolor=white,
	skipbelow=\baselineskip,
	skipabove=\baselineskip
]{mytheorem}

\tikzstyle{nod}= [circle, draw,inner sep=0pt, minimum size=0.5cm] 

\begin{document}
	
	\newtheorem{mytheorem}{Theorem}[section]
	\newtheorem{mylemma}{Lemma}[mytheorem]	

	\newenvironmentx*{solution}[1]{\section*{Problem #1}\addtocounter{section}{1}\setcounter{mylemma}{0}\setcounter{mytheorem}{0}}{}
	\newenvironmentx*{theorem}[1]{\begin{mytheorem}#1\begin{proof}}{\end{proof}\end{mytheorem}}
	\newenvironmentx*{lemma}[1]{\begin{mylemma}#1\begin{proof}}{\end{proof}\end{mylemma}}


	\begin{solution}{17}
		\begin{theorem}{In a planar triangulation let $n_i$ be the number of vertices of degree $i$. Then, \begin{equation*}\sum_{i \in \mathbb{N}} (6 - i)n_i = 12 \end{equation*}.}
			\begin{lemma}{Let $G = (V, E)$ and $G' = (V \cup \{v\}, E \cup E')$ be planar triangulations. Then $|E'| = 3$ and all edges in $E'$ are indicent to $v$.}
			Since any planar triangulation of $n$ vertices and $e_n$ edges satisifes $e_n = 3n - 6$, we see inductively that
			\begin{align*}
				e_n &= e_{n-1} + 3& (n > 3)\\
				e_3 &= 3
			\end{align*}
			
			Since $G'$ has exactly one vertex more than $G$ and both are planar triangulations, $|E'| = 3$.\\
			\ \\
			Next, we will show that the degree of $v$ exceeds or is equal to $3$ and thus, all edges of $E'$ have to be indicent to $v$.

			By \textsc{Kuratowski}, $G'$ is not a topological minor of $K_{3,3}$ or $K_5$ and any planar triangulation is edge-maximal. By \emph{Lemma 4.4.5 (any edge-maximal graph without topological minors $K_{3,3}, K_5$ is $3$-connected)}, $G'$ is $3$-connected.

			If the degree of $v$ deceeded $3$, then $G'$ would not be $3$-connected (it could be isolated by removing two vertices) .
			
			Hence, all three edges of $E'$ are indicent to $v$.
			\end{lemma}
			
			We will show by induction on the number of vertices $n$ of a planar triangulation $G$ with $n_i$ vertices of degree $i$ ($i \in \mathbb{N}$) that 
			\begin{equation*}
				T_G := \sum_{i \in \mathbb{N}} (6 - i)n_i = 12
			\end{equation*}
	.
	
			\begin{itemize}
				\item Base $\mathbf{n = 3}$
					
					Then, the graph is a triangle and the condition is satisifed:
					\begin{align*}
						T_{K_3} = \sum_{i \in \mathbb{N}} (6 - i)n_i = (6-2) \cdot 3 = 4 \cdot 3 = 12
					\end{align*}
				\item Step $\mathbf{n \geq 4}$
					
					Any $n$-vertex planar triangulation $G = (V, E)$ has a subgraph $H = (V', E')$ which is a  ($n - 1$)-vertex planar triangulation.
	
					By \emph{Lemma 1.1.1}, there is a vertex $v \in V \setminus V'$ of degree $3$. Furthermore, the degree of exactly three other vertices $v_i, v_j, v_k \in V$ is increased. Thus $E \setminus E' = \{\{v, v_i\}, \{v, v_j\}, \{v, v_k\}\}$ and for $T_G$:
					\begin{align*}
						T_G=&T_H\\
						 &+ (6 - 3)\\
						 &+ (6 - (d(v_i) + 1)) - (6 - d(v_i)))\\
						 &+ (6 - (d(v_j) + 1)) - (6 - d(v_j))\\
						 &+ (6 - (d(v_k) + 1)) - (6 - d(v_k))\\
						=&T_H +  3 - 1 - 1 -1\\
						=&T_H\\
						=&12&\text{(by induction)}
					\end{align*}
			\end{itemize}
		\end{theorem}
	\end{solution}
	\newpage
	\begin{solution}{18}
	\end{solution}
	\newpage
	\begin{solution}{19}
	\end{solution}
	\newpage
	\begin{solution}{20}
	\end{solution}
	
\end{document}
