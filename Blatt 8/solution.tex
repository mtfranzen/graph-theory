% all-in-one cheatsheet layout (Michael Franzen, 2013)
\documentclass[a4paper]{article}

% geometry settings
\usepackage[top=2cm, bottom=2.5cm, left=2cm, right=2cm]{geometry}

% font settings
%\usepackage[light,math]{kurier}
\usepackage[T1]{fontenc}
\usepackage[utf8]{inputenc}
\usepackage{marvosym}
\usepackage{amssymb}
\usepackage{amsfonts}
\usepackage{amsmath}
\usepackage{amsthm}

% colors
\usepackage{xcolor}
\definecolor{lightgray}{gray}{0.8}

% formatting
\usepackage{paralist}
\usepackage{multicol}
\usepackage{tabularx}
\usepackage{Tabbing}
\usepackage{booktabs}
\usepackage{fancyhdr}
\usepackage{url}
\usepackage[framemethod=tikz]{mdframed}
\pagestyle{fancy}

% math
\usepackage{array}
\usepackage{eqnarray}
\usepackage{mathtools}
\usepackage{cases}

% figures
\usepackage{wrapfig}
\usepackage{subfigure}

% figure modules
\usepackage{graphicx}
\usepackage{tikz}
\usetikzlibrary{positioning,calc, shapes}
\usepackage{algorithm2e}
\usepackage{verbatim}	

% TOC & Glossary
\usepackage{sectsty}
\usepackage[nottoc,notlof,notlot]{tocbibind}
\usepackage[titles,subfigure]{tocloft}

% commands
\usepackage{xargs}
\usepackage{ifthen}

% head line
\fancyhf{}
\chead{Graph Theory - Sheet 8 - \today\\J. Batzill (1698622), M. Franzen (1696933), J. Labeit (1656460)}
\renewcommand{\headrulewidth}{0.4pt} %obere Trennlinie

\newcommand{\sheetnumber}{1}

% (problem number)
\surroundwithmdframed[
    hidealllines=true,
    backgroundcolor=gray!10,
    skipbelow=\baselineskip,
    skipabove=\baselineskip
]{mylemma}

\surroundwithmdframed[
	linecolor=white,
	skipbelow=\baselineskip,
	skipabove=\baselineskip
]{mytheorem}

\tikzstyle{nod}= [circle, draw,inner sep=0pt, minimum size=0.5cm] 

\begin{document}
	
	\newtheorem{mytheorem}{Theorem}[section]
	\newtheorem{mylemma}{Lemma}[mytheorem]	

	\newenvironmentx*{solution}[1]{\section*{Problem #1}\addtocounter{section}{1}\setcounter{mylemma}{0}\setcounter{mytheorem}{0}}{}
	\newenvironmentx*{theorem}[1]{\begin{mytheorem}#1\begin{proof}}{\end{proof}\end{mytheorem}}
	\newenvironmentx*{lemma}[1]{\begin{mylemma}#1\begin{proof}}{\end{proof}\end{mylemma}}


	\begin{solution}{Problem 29}
		\begin{theorem}{For every $k \in \mathbb{N} $ there exists a tree $T_k$ with $\Gamma (T_k) = k$}
			In the following we will show how to construct $T_k$ by induction. 
			Additionally to $\Gamma (T_k) = k$, every $T_k$ will have a greedy coloring so that the root of $T_k$ is colored with color $k$.  
			\begin{itemize}
				\item \emph{Base} \\
					$T_1$ is just a single vertex without edges. Obviously $\Gamma (T_1) = 1$ and the color of the root of $T_1$ is 1.
				\item \emph{Induction step} \\
					For any $k > 1$, $T_k$ can be constructed by using one new vertex $v$ and $T_1$, $T_2$, ... $T_{k-1}$. 
					By connecting the roots of $T_1$, $T_2$ ... $T_{k-1}$ to $v$ we ensure that there is a greedy coloring in which $v$ has to have the color $k$. 
					This greedy coloring can be achieved by first coloring $T_1$, $T_2$, ...$T_{k+1}$ by induction so that the roots have the colors $1$, $2$, ... $k-1$. 
					Now we can color $v$ with the color $k$ because by construction all colors smaller than $v$ are already taken by the nodes adjacent to $v$. 
					Hence, we constructed $T_k$ with a root node $v$ which has color $k$ and $\Gamma (T_k) = k$. 
			\end{itemize}
		\end{theorem}
		Finally, by using the proven theorem we know that $min\{k \in \mathbb{N} | \Gamma(T) \leq k\ for\ all\ trees\ T\} = \infty$. 
		
		
		\begin{theorem}{For any Graph $G$  $\Gamma(G) \leq max_{uv \in E(G)} min\{deg(u), deg(v)\} + 1$}
		For the sake of contradiction let's assume that there is a graph $G$ with $\Gamma(G) = k$ and $k > max_{uv \in E(G)} min\{deg(u), deg(v)\} + 1$. 
		Additionally, let $G$ be colored with a greedy coloring using $k$ colors which obviously has to exist if $\Gamma(G) = k$. \\
		Let $v \in V(G)\ with\ c(v) = k$ be one of the vertices with the highest color. 
		Now, $deg(v) \geq k-1$  for $c(v)=k$ to be possible in a greedy coloring, because $v$ has to have atleast $k-1$ adjacent vertices which are colored in colors $1$ through $k-1$. \\
		By assumption we know that for all neighbours $u$ of $v$, $k> min\{deg(v), deg(u)\} + 1$. 
		Now we easily see that $deg(u) < deg(v)$ because else  $k > min\{deg(v), deg(u)\} + 1 = deg(v) + 1$ but earlier we saw that $deg(v) \geq k-1$. 
		Hence, we know that for all neighbours $u$ of $v$ the following inequality holds: $k > min\{deg(u), deg(v)\} + 1 = deg(u) + 1 \Leftrightarrow deg(u) < k - 1$. \\
		Now for $c(v)=k$ to hold in a proper greedy coloring we need to find a neighbour $u$ of $v$ with $c(u)=k-1$ which was colored before $v$. 
		Because $deg(u) > k-1$ $u$ only has $k-2$ neighbours which are not $v$. 
		Hence if $v$ is not already colored $c(u) < k-1$ so it is impossible to greedy color $v$ with the color $k$. 
		This finally leads to a contradiction, hence $k > max_{uv \in E(G)} min\{deg(u), deg(v)\} + 1$ was a wrong assumption and for every graph $\Gamma(G) > max_{uv \in E(G)} min\{deg(u), deg(v)\} + 1$ holds.
		\end{theorem}
	\end{solution}
	\newpage
	\begin{solution}{Problem 32}
		\begin{theorem}{The adjacency matrix of any $d$-regular graph has an eigenvalue of $d$.}
			Let $G = (V, E)$ be a $d$-regular graph and let $A(G) = (a_{ij})_{i, j = 1,\dots, n}$ ($n \in \mathbb{N}$) denote it's adjacency matrix. We show that
				\begin{equation*}
					A(G) \cdot (1, \dots ,1)^{\top} = (d,  \dots ,d)^{\top} = d \cdot (1,  \dots ,1)^{\top}
				\end{equation*}
			and thereby that $d$ is an eigenvalue of $G$. Since $G$ is $d$-regular, every row sum equals exactly $d$:
				\begin{equation*}
					\forall i \in [1, \dots, n]: \sum_{j=1}^{n} a_{ij} = d
				\end{equation*}
			Trivially,
				\begin{equation*}
					A(G) \cdot (1, \dots ,1)^{\top} =  (\sum_{j=1}^{n} a_{1j},  \dots , \sum_{j=1}^{n} a_{nj})^{\top} = (d, \dots, d)^\top = d \cdot (1,  \dots ,1)^{\top}
				\end{equation*}
		\end{theorem}

		\begin{theorem}{The adjacency matrix of any bipartite, $d$-regular graph has an eigenvalue of $-d$.}
			Let $G = (V, E)$ be a bipartite, $d$-regular graph and let $A(G) = (a_{ij})_{i, j = 1,\dots, n}$ ($n \in \mathbb{N}$) denote it's adjacency matrix.
			
			For all of $G$'s vertices $v_1, ..., v_n$ ($n \in \mathbb{N}$), let $\sigma(v_i)$ denote a selection function that assigns each vertex it's partition:
\[
 \sigma(v) =
  \begin{cases}
   -1 & \text{if } v \text{ in partition \#1}\\
   1 & \text{if } v \text{ in partition \#2}
  \end{cases}
\]
Let $\mathbf{x}$ denote
	\begin{equation*}
		A(G) \cdot (\sigma(v_1), \dots, \sigma(v_n))^\top = \mathbf{x} = (x_1, \cdots, x_n)^\top
	\end{equation*}
We will prove that $(\sigma(v_1), \dots, \sigma(v_n))^\top$ is an eigenvector for the eigenvalue $-d$ by showing that $\mathbf{x} = (-d \cdot \sigma(v_1), \dots, -d \cdot \sigma(v_n))$.\\

For any $i \in \{1, \dots, n\}$,
\begin{equation*}
	x_i := \sum_{j=1}^n a_{ij} \cdot \sigma(v_j)
\end{equation*}
Since $v_i$ is only adjacent to vertices which are not in it's partition and $\sigma(v_i) = -\sigma(v_j)$ if $i, j$ are in different partitions.
\begin{equation*}
	x_i := \sum_{j=1}^n a_{ij} \cdot -\sigma(v_i)
\end{equation*}
Moreover, $G$ is $d$-regular. Thus,
\begin{equation*}
	x_i :=  \sum_{i=1}^d 1 \cdot -\sigma(v_i) = -d \cdot \sigma(v_i) 
\end{equation*}
Hence, $A(G) \cdot (\sigma(v_1), \dots, \sigma(v_n))^\top = \mathbf{x} = -d \cdot (\sigma(v_1) , \dots, \sigma(v_n) )^\top$ and thereby, $-d$ is an eigenvalue of $G$.
		\end{theorem}
	\end{solution}

\end{document}
