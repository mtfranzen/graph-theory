% all-in-one cheatsheet layout (Michael Franzen, 2013)
\documentclass[a4paper]{article}

% geometry settings
\usepackage[top=2cm, bottom=2.5cm, left=2cm, right=2cm]{geometry}

% font settings
%\usepackage[light,math]{kurier}
\usepackage[T1]{fontenc}
\usepackage[utf8]{inputenc}
\usepackage{marvosym}
\usepackage{amssymb}
\usepackage{amsfonts}
\usepackage{amsmath}
\usepackage{amsthm}

% colors
\usepackage{xcolor}
\definecolor{lightgray}{gray}{0.8}

% formatting
\usepackage{paralist}
\usepackage{multicol}
\usepackage{tabularx}
\usepackage{Tabbing}
\usepackage{booktabs}
\usepackage{fancyhdr}
\usepackage{url}
\usepackage{mdframed}
\pagestyle{fancy}

% math
\usepackage{array}
\usepackage{eqnarray}

% figures
\usepackage{wrapfig}
\usepackage{subfig}

% figure modules
\usepackage{graphicx}
\usepackage{tikz}
\usetikzlibrary{positioning,calc}
\usepackage{algorithm2e}
\usepackage{verbatim}

% TOC & Glossary
\usepackage{sectsty}
\usepackage[nottoc,notlof,notlot]{tocbibind}
\usepackage[titles,subfigure]{tocloft}

% commands
\usepackage{xargs}
\usepackage{ifthen}

% head line
\fancyhf{}
\rhead{Michael Franzen (1696933), \today}
\lhead{Graph Theory - solution of problem sheet 1 }
\renewcommand{\headrulewidth}{0.4pt} %obere Trennlinie

\newcommand{\sheetnumber}{1}

% (problem number)
\surroundwithmdframed[
    hidealllines=true,
    backgroundcolor=gray!10,
    skipbelow=\baselineskip,
    skipabove=\baselineskip
]{mylemma}

\surroundwithmdframed[
	linecolor=white,
	skipbelow=\baselineskip,
	skipabove=\baselineskip
]{mytheorem}

\begin{document}
	
	\newtheorem{mytheorem}{Theorem}[section]
	\newtheorem{mylemma}{Lemma}[mytheorem]	

	\newenvironmentx*{solution}[1]{\section*{Problem #1}\addtocounter{section}{1}\setcounter{mylemma}{0}\setcounter{mytheorem}{0}}{}
	\newenvironmentx*{theorem}[1]{\begin{mytheorem}#1\begin{proof}}{\end{proof}\end{mytheorem}}
	\newenvironmentx*{lemma}[1]{\begin{mylemma}#1\begin{proof}}{\end{proof}\end{mylemma}}


	\begin{solution}{1}
		\begin{theorem}{Any tree $T$ has at least $\Delta(T)$ leaves.}
			\begin {lemma}{Any connected subgraph of a tree is a tree as well.}
				If a graph $G = (V, E)$ is acyclic then $E$ does not contain any cyclic subset and hence there is no cyclic subgraph of $G$. From these considerations, any connected subgraph of $G$ is acyclic and therefore a tree.
			\end{lemma}
		
			\begin{lemma}{Any tree $T$ with $\Delta(T)+1$ vertices has $\Delta(T)$ leaves ($\Delta(T) \geq 1$).}
				Let $v_0 \in V_T$ denote the vertex of $T = (V_T, E_T)$ with $d(v_0) = \Delta(T)$.\\
				For any $v, w \in V_T\setminus\{v_0\}$ $(v \neq w)$:
				\begin{itemize}
					\item $d(v) \geq 1$: $v_0$ is adjacent to $v$ since $|V_T| =  \Delta(T)+1$. $\Rightarrow d(v) \geq 1$
					\item $d(v) \leq 1$: Any edge $\{v, w\} \in E_T$ would create a cycle $(v_0, v, w, v_0)$ which would render $T$ invalid as a tree.
				\end{itemize}
				$\Rightarrow \forall v \in V_T\setminus\{v_0\}: d(v) = 1 \Leftrightarrow v \text{ is a leaf}$. Notably, $|V_T\setminus\{v_0\}| = \Delta(T)$.
			\end{lemma}

			\begin{lemma}{For any tree $T =  (V_T, E_T)$ with $|V_T| > \Delta(T)+1 \geq 3$, there exists a partition $(S, S')$  of $T$ with $\Delta(S) + \Delta(S') = \Delta(T)$.}
				Let $v_0 \in V_T$ denote a vertex with $d(v_0) = \Delta(T)$. Since $|V_T| > \Delta(T)+1$, $v0$ has at least one non-leaf adjacent vertex $v_1 \in V_T$. Now, let $(S_0, S_1)$ denote the partition at the edge $\{v_0, v_1\}$ whereby $S_0$ contains $v_0$ and ($S_1$, $v_1$ respectively).\\
				As seen in Lemma 1.1, $S_0$ and $S_1$ are trees.
				\begin{itemize}
					\item case $d(v_1) = 2$: In this case, a cut at the edge $\{v_0, v_1\}$ would make 
				\end{itemize}
			\end{lemma}
			\ \\
			\emph{Proof by induction.} Let $T = (V_T, E_T)$ be a tree.\\ \ \\
			\textbf{Basis: }$\Delta(T)=1$\\
				Inherently, $V_T =^! \{v_1, \dots, v_m\}$ $(v_i = v_j \Rightarrow i=j)$ and $E_T =^! \{\{v_1, v_2\}, ..., \{v_{m-1}, v_m\}\}$ $(m,i,j \in \mathbb{N})$. Any modification would violate our preconditions. $v_1, v_m$ are the only vertices with degree $1$. Therefore, the number of leaves is $2 \geq \Delta(T)$. \ \\
			\ \\
			\textbf{Step: }For some $n \in \mathbb{N}: \text{ Any } T = (V_T, E_T) \text{ with } \Delta(T) = n \text{ has at least } n \text{ leaves}$ .\ \\
			Let $T' = (V_{T'}, E_{T'})$ be a tree with $\Delta(T') > n$.
		\end{theorem}
	\end{solution}
	\newpage
	\begin{solution}{2}
		\begin{theorem}{If any removal of an edge increases the number of connected components of a graph $G$ then $G$ is acyclic.}
			Let $S = (V_S, E_S)$ be a connected component of $G$. Removing any edge $e \in E_S$ increases the number of connected components and thereby disconnects $S$.\\
			\textbf{If $S$ had contained a cycle $C$ then there would not have been any unique paths in $C$}. Hence, no edge removal would have an effect on the connectivity of $S$ (removing an edge still leaves a path between any two vertices). Hereby, $S$ had to be acyclic.\\
			 If any connected component of $G$ was acyclic then $G$ would be acyclic, too.
		\end{theorem}
		\begin{theorem}{If adding any edge introduces a cycle in an acyclic graph $G = (V, E)$ then any two vertices in $G$ are joined by a unique path.}
			If adding an edge $\{v_0, v_1\}$ joining two non-adjacent vertices $v_0, v_1 \in V$ introduces a cycle $(v_0, ..., v_1, v_0)$ then there had to be \emph{at least} one path from $v_0$ to $v_1$.\\
			Furthermore, if there was \emph{more than one} path joining $v_0$ and $v_1$ then there would have already been a cycle (but $G$ is acyclic). $\Rightarrow$ Any vertex had to be joined by a unique path.
		\end{theorem}
		\begin{theorem}{If any two vertices in a graph are joined by a unique path then any removal of an edge increases the number of connected components.}
			Let $G = (V, E)$ be a graph in which all vertices are joined by a unique path. Let $e = \{v_0, v_1\} \in E$ be an edge. Thus, the unique path from $v_0$ to $v_1$ runs over (and is exactly) $e$. From these considerations, removing $e$ would make $v_1$ inaccessible from $v_0$ and would thereby increase the number of connected components.
		\end{theorem}
	\end{solution} 
	\newpage
	\begin{solution}{3}
		\begin{theorem}{Either a graph or its complement is connected.}
			If $G$ was connected, we would already been done.\\
			Let $G = (V, E)$ be disconnected. Moreover, let $S = (V_S, E_S)$ and $T = (V_T, E_T)$ be two connected components of G ($S \neq T$).\\
			Now let $u, v \in V_S \cup V_T$ be vertices in these components.
			\begin{itemize}
				\item \textbf{Case 1: } $u \in V_S$ and $v \in V_T$.\\
					Then $G$ does not contain the edge $e = \{u, v\}$ . Otherwise, S and T were interconnected. From these considerations, the graph's complement $\bar{G}$ does contain $e$.
				\item  \textbf{Case 2: } $u, v \in V_S$:\\
					Considering that $V_T$ is not empty, then there is a vertex $w \in V_T$ such that the edges $\{u, w\}$ and $\{v, w\}$ exist in $\bar{G}$ (see Case 1). From these considerations, $\bar{G}$ also contains the path $(u, w, v)$ .
			\end{itemize}
			Therefore, all vertices in a \emph{disconnected graph} $G$ are connected in $\bar{G}$ and hereby $\bar{G}$ is connected.\\
		\end{theorem}
	\end{solution} 
	\newpage
	\begin{solution}{4}
		\begin{theorem}{If $u$ and $v$ are the only vertices of odd degree in a graph then there is a $u$-$v$-path.}
			
		\end{theorem}
	\end{solution} 
\end{document}